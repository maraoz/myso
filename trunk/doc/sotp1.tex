\documentclass[a4paper,10pt]{elsart}
\usepackage[pdftex,usenames,dvipsnames]{color}
\usepackage [spanish] {babel}
\usepackage[utf8]{inputenc}
\usepackage{amsmath, amsfonts, amssymb}
\usepackage[pdftex]{color,graphicx}
\include{latexsym}
\include{epsf}

\begin{document}


\begin{frontmatter}
\title{Trabajo Práctico 1: Filesystems, IPCs, Servidores}
\author{Manuel Aráoz (49244) \& Matías Williams (49195)} \& Pablo Giorgi (49222)}
\address{Ingeniería en Informática\\
Instituto Tecnológico Buenos Aires\\
Ciudad Autónoma de Buenos Aires, Argentina}
\date{\today}


\maketitle

\begin{abstract}
En este informe se darán a conocer las decisiones que se tomaron al momento de realizar el Trabajo Práctico, los problemas presentados y las soluciones que se aplicaron para resolverlos.
\end{abstract}
\begin{keyword}
Programación lineal, programación entera, método Símplex, método de Resolución Gráfica, Camino hamiltoneano.
\end{keyword}
\end{frontmatter}
\clearpage

\tableofcontents
\listoffigures

\clearpage

\section{Divisón en Capas}

\subsection{Aplicación}
La capa de aplicación consiste en distintos modulos. Por un lado existen los módulos \emph{maincore.c} y \emph{mainlines.c}. Estos módulos son los que contienen los \emph{main} de los procesos que van a estar corriendo a la vez. Solo habrá un proceso corriendo \emph{maincore.c} y habrá tantos procesos como archivos haya en el directorio de lineas, corriendo \emph{mainlines.c}. También, existen otros módulos que complementan las funciones de los dos \emph{main}: \emph{lineas.c} y \emph{core.c}. Aquí se hallan todas las funcionen que se van a llamar desde los dos \emph{main}. Asimismo, hay un módulo que se encarga de la interfaz gráfica: \emph{draw.c}.
\subsubsection{Decisiones tomadas}
\subsubsection{Problemas presentados}
\begin{itemize}
	\item Al principio no se chequeaba que un colectivo insertado no se pudiera insertar nuevamente. Esto generaba dos problemas; por un lado se insertaba dos veces el mismo colectivo, y por el otro varias veces se quería mover un colectivo que no estaba insertado. El segundo problema surgió porque se quiso arreglar el primero mediante utilizar un valor inválido para.
\end{itemize}


\subsubsection{Soluciones a los problemas}
\subsection{Marshall}
\subsubsection{Decisiones tomadas}
\subsubsection{Problemas presentados}
\subsubsection{Soluciones a los problemas}
\subsection{Transporte}
\subsubsection{Decisiones tomadas}
\subsubsection{Problemas presentados}
\subsubsection{Soluciones a los problemas}

\section{Bibliografía}
\begin{itemize}
	\item GROSSMAN, Stanley I., \emph{Aplicaciones del Álgebra Lineal}, McGraw-Hill, Cuarta Edición, México D.F., 1992.
	\item MUNIER, Nolberto J., \emph{Programación Lineal}, Editorial Astrea, Tercera Edición, Buenos Aires, 1979.
	\item MUNIER, Nolberto J., \emph{Aplicaciones de la Programación Lineal}, Editorial Astrea, Buenos Aires, 1986.
	\item WANER, Stefan, \emph{Simplex Method Tool},2008, \underline{http://www.zweigmedia.com/RealWorld/simplex.html}
\end{itemize}


\end{document}

